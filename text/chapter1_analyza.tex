\chapter{Analýza}

\section{Stupne Napredovania}
Stupne napredovania sú programovou časťou slovenského skautingu. Tvoria základný kameň programu a osobného rastu. Sú určené pre deti a dospievajúcich od 4 rokov až po 24 rokov. Program je rozčlenený do nasledujúcich vekových kategórií:
\begin{description}
    \item[4 - 6 rokov] Janík a Grétka
    \item[7 - 10 rokov] Vlčia stopa
    \item[11 - 14 rokov] Skautský chodník
    \item[15 - 18 rokov] Rangerský horizont
    \item[19 - 24 rokov] Roverský svet
    \item[11+ rokov] Nováčik - pre všetkých nových členov v Slovenskom skautingu.
\end{description}

\subsection{Skautský chodník}
Skautský chodník sa delí na tri časti. Prvý skaut, Neznáme cesty a Posledný vrchol. Každá z týchto troch častí by mala zabrať rok.

Skautský chodník sa zameriava na komplexný rozvoj jedinca naprieč rôznymi oblasťami. Pokrýva rozvoj zo svojho vnútra cez duchovný, citový a intelektuálny rozvoj, k rozvoju svojho vonkajšku pomocou telesného rozvoju až ku širšiemu sociálnemu rozvoju a buduje charakter. Pre jednoduchšie uchopenie je tento rozvoj koncipovaný do 4 kategórií.

Charakter (modrá) je kategória zameraná na rozvoj osobnosti a hodnôt. Hlavnou súčasťou tejto kategórie je sociálny aspekt a to najmä v rámci družiny a oddielu. 

Služba (červená) je kategória zameraná na ekológiu, rodinu a život v prírode a meste, spoznávanie rozmanitosti iných kultúr, zlepšovanie komunikačných schopností a duchovného dedičstva skautingu.

Zdravie a sila (žltá) sa zaoberá fyzickou kondíciou a životom na tábore.

Schopnosti a zručnosti (zelená) je dedikovaná intelektuálnemu rozvoju a základným schopnostiam ktoré by mal poznať každý ako napríklad varenie.

\subsection{Skautský chodník}
Skautský chodník sa delí na tri časti. Prvý skaut, Neznáme cesty a Posledný vrchol. Každá z týchto troch častí by mala zabrať rok.

Skautský chodník sa zameriava na komplexný rozvoj jedinca naprieč rôznymi oblasťami. Pokrýva rozvoj zo svojho vnútra cez duchovný, citový a intelektuálny rozvoj, k rozvoju svojho vonkajšku pomocou telesného rozvoju až ku širšiemu sociálnemu rozvoju a buduje charakter. Pre jednoduchšie uchopenie je tento rozvoj koncipovaný do 4 kategórií.

Charakter (modrá) je kategória zameraná na rozvoj osobnosti a hodnôt. Hlavnou súčasťou tejto kategórie je sociálny aspekt a to najmä v rámci družiny a oddielu. 

Služba (červená) je kategória zameraná na ekológiu, rodinu a život v prírode a meste, spoznávanie rozmanitosti iných kultúr, zlepšovanie komunikačných schopností a duchovného dedičstva skautingu.

Zdravie a sila (žltá) sa zaoberá fyzickou kondíciou a životom na tábore.

Schopnosti a zručnosti (zelená) je dedikovaná intelektuálnemu rozvoju a základným schopnostiam ktoré by mal poznať každý ako napríklad varenie.
\section{Prípady použitia}

\newlist{usecases}{enumerate}{1}
\setlist[usecases]{label=UC-\arabic*:}

\newcommand{\usecase}[3]{
    \item \textbf{#2}\\
    Aktér: #1\\
    #3
}

\begin{usecases}
    \usecase{Skaut}{Moje napredovanie}{
        Po otvorení aplikácie sa zobrazia úlohy ktoré si použivateľ zvolil ako sledované.
        Zo zoznamu úloh si môže pridať nové úlohy medzi sledované.
        Po dokončení úlohy sa mu úloha odstráni zo sledovaných.
    }
    \usecase{Skaut, Radca}{Vlastná úloha}{
        Skaut zvolí úlohu ktorú chce nahradiť, navrhne nové znenie, úplne nové alebo úprava stávajúceho, a pošle ho radcovi na schválenie.
        V procese schvaľovania môžu ešte znenie novej úlohy pozmeniť. 
        Keď úloha zodpovedá očakávanej náročnosti, tak ju schváli a skautovi sa pridá medzi aktívne úlohy.
    }
    \usecase{Skaut, Radca}{Splnenie úlohy}{
        Po splnení úlohy môže nastať schvaľovanie splnenia zo strany radcu.
        Radca môže zadať hromadné splnenie úlohy členom svojej družiny.
    }
    \usecase{Radca}{Moja družina}{
        Používateľovi sa zobrazí postup všetkých členov jeho družiny.
    }
    \usecase{Zborový}{Môj zbor}{
        Používateľovi sa zobrazia všetci členovia jeho zboru a môže spravovať ich priradenie do družín. Nastavuje Družinových radcov.
    }
    \usecase{Programový}{Úprava programu}{
        Pri zmenách vo vdelávacio výchovnom programe môže programový vedúci zmeniť, pridať alebo odstrániť časti programu. Prípadne môže vytvoriť úplne nový program alebo zrušiť už existujúci.
    }
\end{usecases}

\section{Zber požiadaviek}
Zber požiadaviek 