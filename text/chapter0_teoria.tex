\chapter{Teoretický základ}
Táto kapitola sa venuje predstaveniu prostredia Slovenského skautingu. Následne opisuje aplikácie zaoberajúce sa podobnou tématikou v prostredí iných krajín. Ďalej sú predstavené moderné technolólgie a metódy ktoré sú používané na vývoj aplikácií. 
% predstav slovensky skauting
% resers podobnych aplikacii
% SDLC
% requirements gathering
% laptop vs smartphone
% data synchronization
% usability testing

\section{Slovenský skauting}
Slovenský skauting (ďalej SLSK) je dobrovoľná, nezávislá, nepolitická, nezisková, výchovná organizácia mladých ľudí, prístupná všetkým bez rozdielu pohlavia, pôvodu, národnosti, rasy alebo náboženstva, ktorí chcú žiť a konať v súlade s poslaním, princípmi a výchovnou metódou stanovenými zakladateľom hnutia Róbertom Baden-Powellom. 
SLSK pôsobí najmä v oblasti výchovy detí a mládeže, neformálneho vzdelávania detí a mládeže, vzdelávania dobrovoľných a profesionálnych pracovníkov s mládežou, dospelých a mladých lídrov, občianskej participácie mladých ľudí na živote spoločnosti, cieľavedomej výchovy mladej generácie k demokracii, prevencie pred vznikom závislostí, ochrany životného prostredia, sociálnych aktivít a práce so znevýhodnenými skupinami obyvateľstva, podpory dobrovoľníctva mladých ľudí a voľnočasových aktivít pre mladých ľudí.

\section{Program SLSK}
Program v Slovenskom skautingu je zameraný na tvorbu odboriek, výziev, voľných programových modulov a príručiek osobného napredovania, ktoré vedúci využívajú na pravidelných stretnutiach s našimi členmi.
Týmito aktivitami podporujeme naše vzdelávacie aktivity. 
V roku 2024 sa organizácia v oblasti programu zamerala na dotlač a tvorbu nových publikácií, organizáciu programových akcií a prieskum o budúcom smerovaní a činnosti Programovej rady, ktorá koordinuje a zabezpečuje programové aktivity.

\subsection{Stupne Napredovania}
Stupne napredovania sú programovou časťou slovenského skautingu. Tvoria základný kameň programu a osobného rastu. Sú určené pre deti a dospievajúcich od 4 rokov až po 24 rokov. Program je rozčlenený do nasledujúcich vekových kategórií:
\begin{description}
    \item[4 - 6 rokov] Janík a Grétka
    \item[7 - 10 rokov] Vlčia stopa
    \item[11 - 14 rokov] Skautský chodník
    \item[15 - 18 rokov] Rangerský horizont
    \item[19 - 24 rokov] Roverský svet
    \item[11+ rokov] Nováčik - pre všetkých nových členov v Slovenskom skautingu.
\end{description}

\subsubsection{Skautský chodník}
Skautský chodník sa delí na tri časti. Prvý skaut, Neznáme cesty a Posledný vrchol. Každá z týchto troch častí by mala zabrať rok.

Skautský chodník sa zameriava na komplexný rozvoj jedinca naprieč rôznymi oblasťami. Pokrýva rozvoj zo svojho vnútra cez duchovný, citový a intelektuálny rozvoj, k rozvoju svojho vonkajšku pomocou telesného rozvoju až ku širšiemu sociálnemu rozvoju a buduje charakter. Pre jednoduchšie uchopenie je tento rozvoj koncipovaný do 4 kategórií.

Charakter (modrá) je kategória zameraná na rozvoj osobnosti a hodnôt. Hlavnou súčasťou tejto kategórie je sociálny aspekt a to najmä v rámci družiny a oddielu. 

Služba (červená) je kategória zameraná na ekológiu, rodinu a život v prírode a meste, spoznávanie rozmanitosti iných kultúr, zlepšovanie komunikačných schopností a duchovného dedičstva skautingu.

Zdravie a sila (žltá) sa zaoberá fyzickou kondíciou a životom na tábore.

Schopnosti a zručnosti (zelená) je dedikovaná intelektuálnemu rozvoju a základným schopnostiam ktoré by mal poznať každý ako napríklad varenie.

Každá kategória obsahuje kapitoly ktoré predávajú čitateľovi myšlienku a snažia sa ho niečo nové naučiť. Záverom kapitoly je sada úloh ktoré musí skaut splniť na to aby splnil celú kategóriu. V prípade že by skautovi prišli úlohy moc náročné alebo nemožné splniť, môže požiadať radcu družiny alebo vedúceho o upravenie danej úlohy. Úprava úlohy by mala zachovať obtiažnosť a tému pôvodnej úlohy.

Vyhodnocovanie úloh záleží od jednotlivých úloh. Existujú tri druhy overenia splnenia úlohy. Overenie radcom družiny alebo vedúcim, overenie rodičom skauta a overenie skautom samotným pri čom sa spolieha na jeho skautskú česť. Na úlohách sa cení snaha viac ako samotný výsledok úlohy.

Po splnení určitého množstva úloh sa odomyká skautovi možnosť plniť výzvu z danej kategórie. Každá kategória má vlastnú výzvu.

\subsubsection{Prvý skaut}
Prvý skuat predstavuje prvú časť skautského chodníka. Očakávaný vek skauta ktorý plní Prvého skauta je medzi 11 a 12 rokov. Veku skauta odpovedá aj forma ktorou je vedený Prvý skaut. V rámci Prvého skauta sú všetky úlohy predom dané a skaut nemá žiadnu voľbu čo by plnil. Kapitoly obshaujú veľa rozprávania a témam sa venujú skôr povrchne. Výzvy sa odomykajú po splnení všetkých úloh v danej kategórii.

\subsubsection{Neznáme cesty}
Neznáme cesty predstavujú druhú časť skautského chodníka na ktorý nadväzujú. Oproti Prvému skautovi poskytujú skautovi väčšiu voľnosť vo výbere úloh. Každá úloha má počet bodov ktoré skaut dosiahne za jej splnenie. Cieľom každej kategórie je nazbieranie potrebného počtu bodov z úloh z danej kategórie. Každá kapitola obshuje jednu povinnú úlohu a sadu nepovinných úloh z ktorých si môže skaut vybrať. Výzvy sa odomykajú po splnení povinných úloh z danej kategórie.

\subsubsection{Posledný vrchol}
Posledný vrchol je posledná a záverečná časť skautského chodníka. Predstavuje najväčšiu volňnosť vo výbere úloh z celého skautského chodníka v podobe povinných a voliteľných kapitól. Obsahom kapitol je podobná Neznámym cestám. Na rozdiel od Neznámych ciest v rámci Posledného vrcholu stačí skutovi nazbierať celkový počet naprieč všetkými kategóriami. Má takto najväčšiu volňnosť vo výslednej podobe jeho priechodu Posledným vrcholom.

\subsection{Rangerský horizont}
Rangerský horizont je koncipovaný na maximálne 3 roky. Skladá sa z piatich častí. Osobný rozvoj I, osobný rozvoj II, expedícia I, expedícia II a rangerský projekt. Osobný rast I a II predstavuje podobnú sériu úloh s akou sa skaut stretol počas plnenia skautského chodníka. Slúži na ďalšie prekonávanie samého seba. Expedícia sa zameriava zas na spoznávanie okolia. Na splnenie expedície si je treba vybrať druh expedície na ktorý sa skuat vydá, xy pre I a za pre II. Projekt je určený na tímové riešenie danej problematiky ktorú si rangery zvolia a schváli im ju vodca.


\subsection{Odborky}
Ďalšou programovou časťou sú odborky, ktoré fungujú nezávisle na stupňoch napredovania. Odborky predstavujú súbory úloh, ktorých splnenie vedie k nadobudnutiu odbornosti v danej téme. Delia sa do dvoch stupňov, zelený a červený. Pokiaľ odborka obsahuje zelený stupeň, treba ho získať pred splnením červeného stupňa. Odborky sú podobne ako stupne napredovania rozdelené do viacerých vekových kategórií. Sú prispôsobené charakteristikám a potrebám vekovej kategórie.

\subsection{Výzvy}
Výzvy sú programovou časťou zameranou na prekonávanie samého seba. Vedú skuatov mimo ich komfortnú zónu a vystavujú ich situáciam ktoré nie sú bežné. Výzvy obsahujú ciele ktoré by mali skauti naplniť. Trvanie výziev je rôzne, pri niektorých sa jedná o jednorázovú aktivitu a pri iných zas o dlhodobejšie venovanie sa danej tématike.

% https://www.skauting.sk/skauti/program/vyzvy/

\subsection{Voľné programové moduly}
Dobrovoľným rozšírením programu sú voľné programové moduly. Formou aj obsahom sa najviac približujú ku výzvam.

\subsection{Najvyššie programové ocenenia}
Vyvrcholením programovej činnosti pre vekovú kategóriu je najvyššie programové ocenenie, ktoré sa získava po splnení ostatných častí programu pre danú vekovú kategóriu. 


\section{Analýza podobných riešení}
Celosvetovosť skautského hnutia dáva priestor na vznik viacerým organizáciam zaoberajúcich sa podobnou tématikou. Hľadanie už existujúcich riešení som realizoval vyhľadávaním na Google Play pre operačný systém android a na App Store pre iOS. Vyhľadávanie bolo pomocou kľúčových slov Scout Výsledkom  Jednou z najväčších organizácií v rámci hnutia je Boy Scouts of America. BSA obsahuje podobnú základnú ponuku ako SLSK v oblasti odboriek a najvyších programových ocenení. Na zobrazenie tejto ponuky existuje viacero aplikácií medzi inými Path to Eagle, ScoutChamp.

% https://play.google.com/store/apps/details?id=scoutsandguides.scouts_and_guides&hl=en
% https://play.google.com/store/apps/details?id=com.newryscouts.scoutskills&hl=en
% https://play.google.com/store/apps/details?id=com.widesolutions.sidi&hl=en
% https://apps.apple.com/cz/app/trooptrack-mobile/id1002622052
% https://apps.apple.com/cz/app/troopwebhost/id1524044969
% 

\subsection{Path To Eagle}
Path to Eagle je aplikácia na 

\subsection{ScoutChamp}

\section{Softvérové inžinierstvo}

\subsection{Životný cyklus vývoja softvéru SDLC}
Výber správnej metodológie pre SDLC kladie dôležitú rolu na úspech projektu. SDLC metodológie ponúkajú kostru ktorá sprevádza projekt naprieč návrhom, vývojom a implementáciou softvérového riešenia. SDLC je proces ktorý načŕta fázy a aktivity zapojené do tvorby softvéru od kenceptu cez development po údržbu. Medzi typické fázy SDLC patrí:
\begin{description}
    \item[Plánovanie] definuje rozsah a požiadavky projektu, indetifikuje kľúčových členov a ich role a vytvára plán projektu.
    \item[Analýza] získava a okumentuje podrobné požiadavky od používateľov a kľúčových členov. Analyzuje zozbierané potreby na pochopenie funkčných a nefunkčných požiadavkov systému.
    \item[Dizajn] vytvára architektúru aplikácie, definujúc vzájomné interakcie komponent. Navrhuje podobu používateľského rozhrania a používateľských skúsenosti.
    \item[Implementácia] obsahuje vlastnú tvorbu kódu podľa výstupov z predchádzajúcich fáz napr., požiadavky, návrh UI a ďalších.
    \item[Testovanie] prebieha viacerých úrovniach od testovania jednotlivých častí až po systémové testy, ktoré testujú systém ako celok. V rámci testovania sa verifikuje splnenie požiadaviek.
    \item[Nasadenie] prebieha primárne v dvoch fázach, nasadenie do testovacieho prostredia a do ostrého prostredia. V rámci nasadenia sa rieši finálne prostredie v ktorom budú softvér využívať používatelia.
    \item[Údržba a podpora] sa zameriava na monitorovanie a údržbu softvéru v produkčnom prostredí. Prebiehajú v nej opravy na chyby ktoré sa nenašli v rámci predchádzajúcich fáz. Prípadne sem môžu patriť vylepšenia na základe spätnej väzby od používateľov.
\end{description}

Metodológie SDLC sa líšia primárne v prístupe k týmto fázam. Medzi najnámejšie patrí vodopádová metóda, špirálová metóda, iteratívna metóda, inkrementálna metóda, prototypová metóda, metóda v tvare V, metóda rýchlej aplikácie a agilné metódy.

Vodopád je typický lineárnym priechodom naprieč fázami, kde každá fáza začína po skončení predchádzajúcej. Tento prístup je vhodný na projekty s dobre definovanými požiadavkami na začiatku projektu, a prápadné zmeny sú len minimálné. Pevná štruktúra tejto metódy sa ťažko prispôsobuje zmenám po skončení fáze. Výhodou tejto metódy je dobrá odhadnuteľnosť prostriedkov a času na projekt.

Špiralová metóda rozdeľuje projekt do série iterativných cyklov. V každom cykle sa postupne prejde cez všetky fáze. Na konci cyklu sa spraví vyhodnotenie a spolu so spätnou väzbou vstupujú do plánovania ďalšieho cyklu. Toto umožňuje vyššiu flexibiltu na zmeny. Hlavnou vlastnosťou špirálového modelu je jeho zemranie sa na analýzu rizík. Nevýhodou špirálového modelu je jeho komplexnosť a náročnosť na prostriedky. Zároveň mu chýbajú dobre stavené ciele oproti vodopádovej metóde.

Itertívna metóda sa zameriava na postupný vývoj a zdokonalenie softvéru pomocou iterativných cyklov. Umožňuje kontinuálne zlepšovanie a adaptáciu na zmenu požiadavkov. Výhodou je vytvorenie čiastočného no funkčného produktu na konci každej iterácie. Nevýhodou však môže byť prerastanie rozsahu projektu počas vývojového procesu čo môže viesť k prekročeniu rozpočtu a neskorým časom dodania.

Inkrementálna metóda rozdelí na začiatku projekt na ucelené celky. Tieto celky sú následne vývíjané nezávisle a inrementy sú sekvenčne integrované do projektu. Hlavnou výhodou je možnosť paralalelizmu, no toto zároveň prináša väčsie nároky na koordinovanie tímu. Táto metóda je vhodná len na projekty ktoré umožňujú rozdelenie na nezávislé časti.

Prototypová metóda je charakterizovaná čo najrýchlejším vytvorením prototypuna získanie spätnej väzby od používateľov. Prototypová metóda obsahuje takisto viacero iterácií, kde spätná väzba z predchádzajúcej iterácie pomáha dotvárať požiadavky na ďalšiu iteráciu. Pravidelná spätná väzba prispieba k zlepšeniu komunikácie medzi vývojovým tímom a koncovými používateľmi. Medzi nevýhody patrí náročný odhad rozpočtu a dĺžky práce, s neznámim počtom iterácií vopred. Zároveň môžu používatelia nadobudnúť pocit že sa jedná o hotový systém aj keď v realite ide iba o prototyp.

Metóda v tvare V je nadstavbou nad klasickou vodopádovou metódou. Oproti vodopádovej metóde obsahuje proces verifikácie a validácie ktorý je aktívny paralelne s hlavnými fázami vodopádovej metódy. Každá vývojová fáza má prislúchajúcu testovaciu fázu a až po skončení oboch sa prechádza do ďalšej fáze. Takto nastavený proces znižuje riziko vynechania nejakej veci pred pokračovaním vo vývoji.

Metóda rýchlej aplikácie je druh inkrementálnej metódy so zameraním sa na čo najrýchlejšie vytvorenie aplikácie. Často sú používané nástroje na tvorbu kódu a vlastný kód je držaný na minime. Zároveň sa zameriava na modularizáciu softvéru a paralelný vývoj nezávislých modulov čo rapídne znižuje dobu vývoja. Táto metóda je vhodná pre väčšie tími kde je dostatok vývojarov na paralelný vývoj softvéru.

Agilná metodológia reprezentuje rodinu iteratívnych a inkrementálnych metód ktoré prioritizujú adaptabilitu a sploprácu so zákazníkom. V základe rozdeľuje projekt na šprinty. Jeden šprint trvá typicky 1 až 4 týždne a predchádza mu scrum, čo je stretnutie na synchronizáciu vykonávanej práce. Agilný prístup zdieľa väčšinu výhod a nevýhod z ostatných iteratívnych metód.



\subsection{Zber požiadaviek}
Softvérové požiadavky predstavujú spôsob akým môžme zachytiť potreby zákazníka. Upresňujú nám rozsah projektu ako do funkcionilty ktorú zákazník očakáva, tak aj do formy akou budú tie funkcionality využité. Zároveň nám môžu upresňovať rozsah aj z opačnej strany v zmysle funkcionalít a vlastností ktoré softvér obsahovať nebude. Dotvára to celkový obraz zákazníkovi a predchádza sa pomocou toho nedorozumeniam. 

Dôležitou úlohou na začiatku projeku je získavanie a dokumentácia požiadaviek. Získavanie požiadaviek je proces ktorý prebieha v kooperácií so zákazníkom, kľúčovými členmi a používateľmi. Existuje viacero spôsobom ktorými sme schopní zozbierať tieto požiadavky.

Rozhovory jeden na jednoho

Skupinové rozhovory

Brainstorming

Cieľová skupina

Dotazník

workshop požiadaviek

Sledovanie používateľa

Analýza rozhrania

Analýza dokumentov

Spätné inžinierstvo

Vytváranie prototypov



% https://www.indeed.com/career-advice/career-development/requirement-gathering-techniques
% https://medium.com/@marcus_davis/reliable-requirement-gathering-techniques-in-system-analysis-and-design-you-should-know-c28928dfced1

\subsubsection{Kategorizácia požiadaviek}
Po zozbieraní požiadviek, prichádza na rad rozdelenie požiadaviek podľa rôznych metrík. Hlavnými metrikami zvyknú bývať dôležitosť požiadavky, na ktorú sa používa kategorizácia MoSCoW. Táto kategorizácia delí požiadavky na tie ktoré musia byť zapracované do softvéru, teda must have, tie ktoré by mali byť zapracované do systému, teda should have, tie ktoré by mohli byť vo výslednom softvéri, teda could have a do poslednej kategorizácie spadajú tie ktoré nebudú implementované do softvéru, teda won't have.

Ďalšou metrikou je FURPS+, ktoré delí požiadavky na funkčné, používateľné, spoľahlivé, výkonostné a 


\section{Webové inžinierstvo}

\subsection{Klient-server architektúra}
Klient-server architektúra je sieťový model v ktorom dve hlavné entity, klienti a server, komunikujú medzi sebou za účeľom splenenie špecifickych úloh alebo zdieľania dát. Klient iniciuje požiadavku, čaká na odpoveď serveru a následne zobrazí dáta používateľovi. Server spracúva požiadavky, získa potrebné dáta a pošle ich klientovi. 

Výhodou tejto architektúry je centralizovaná správa prostriedkov a dát. Táto výhoda sa zároveň stáva aj nevýhodou v podobe jediného bodu zlyhania. V prípade zlyhania serveru dochádza k nedostupnosti prostriedkov a dát.

Architekúra môže byť rozšírená o ďalšie prvky medzi serverom a klientom v podobe middlewaru.
%https://medium.com/nerd-for-tech/client-server-architecture-explained-with-examples-diagrams-and-real-world-applications-407e9e04e2d1


%https://www.coursera.org/articles/client-server-architecture
%https://www.ebsco.com/research-starters/architecture/client-server-architecture


\subsection{API}
API, aplikačné programovacie rozhranie slúži na komunikácie medzi programami. 
%https://www.ibm.com/think/topics/api
% https://aws.amazon.com/what-is/api/
% https://www.oracle.com/europe/cloud/cloud-native/api-management/what-is-api/

\subsubsection{SOUP API}


\subsubsection{RESTful API}


\subsubsection{Graph API}


\subsubsection{gRPC API}

\subsection{OAuth}

\subsection{Ako služba}

\subsubsection{Typy služieb}

\subsection{Backend ako služba}



\section{Tvorba použivateľského rozhrania}



\subsection{Základné prvky použivateľského rozhrania}
Štvrtou heuristikou Jakoba Nielsna je udržanie konzistencie a štandardy pri návrhu používateľského rozhrania.
Z tohto dôvodu ustálili základné prvky, ktoré sa používajú v rámci používateľského rozhrania.

%https://www.nngroup.com/articles/ui-elements-glossary/


\subsection{Prototyp}
Prototyp slúži na otestovanie použiteľnosti použivateľského rozhrania pred tým ako investujeme prostriedky do finálneho riešenia.
Podľa vernosti s ktorou sa podobajú na finálny produkt rozlišujeme dve kategórie prototypov. Vernosť môže byť v rôznych oblastiach interaktivita, vizuály alebo obsah a príkazy.
Existujú dva druhy prototypov podľa množstva prostriedkov potrebných na ich zostavenie.

\begin{description}
    \item [Prototyp s nízkou vernosťou] {
        umožňuje jednoduché úpravy vďaka jeho jednoduchosti. Jednoduchosť prototypu umožňuje predstavenie základnej štruktúry používateľského rozhrania. Na druhú stranu tento prototyp neposkytuje žiadne takmer žiadnu interaktivitu pre používateľa a všetky prechody sa dejú počas testovania manuálne. 
    }
    \item [Prototyp s vysokou vernosťou] {
        poskytuje reálnejšie odozvy systému. Pri testovaní prototypu s vysokou vernosťou môže nastať nedorozumenie, keď sa môže výsledný systém zdať pripravený na použitie aj keď sa jedná len o prototyp.
    }
\end{description}
% https://www.nngroup.com/articles/ux-prototype-hi-lo-fidelity/



\subsection{Persóny}
Persońa je fiktívny, ale realistický popis typického používateľa produktu.


% https://www.nngroup.com/articles/personas-study-guide/



\section{Testovanie použivateľského rozhrania}


\subsection{Testovania mladistvým}
https://www.nngroup.com/articles/usability-testing-minors/
https://www.nngroup.com/articles/usability-of-websites-for-teenagers/
https://www.nngroup.com/articles/childrens-websites-usability-issues/
https://www.nngroup.com/articles/kids-cognition/

