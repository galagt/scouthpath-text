\chapter{Teoretický základ}

\section{Softvérové inžinierstvo}

\subsection{Životný cyklus vývoja softvéru SDLC}


\subsection{Zber požiadaviek}



\section{Webové inžinierstvo}

\subsection{Klient-server architektúra}
Klient-server architektúra je sieťový model v ktorom dve hlavné entity, klienti a server, komunikujú medzi sebou za účeľom splenenie špecifickych úloh alebo zdieľania dát. Klient iniciuje požiadavku, čaká na odpoveď serveru a následne zobrazí dáta používateľovi. Server spracúva požiadavky, získa potrebné dáta a pošle ich klientovi. 

Výhodou tejto architektúry je centralizovaná správa prostriedkov a dát. Táto výhoda sa zároveň stáva aj nevýhodou v podobe jediného bodu zlyhania. V prípade zlyhania serveru dochádza k nedostupnosti prostriedkov a dát.

Architekúra môže byť rozšírená o ďalšie prvky medzi serverom a klientom v podobe middlewaru.
%https://medium.com/nerd-for-tech/client-server-architecture-explained-with-examples-diagrams-and-real-world-applications-407e9e04e2d1


%https://www.coursera.org/articles/client-server-architecture
%https://www.ebsco.com/research-starters/architecture/client-server-architecture


\subsection{API}
%https://www.ibm.com/think/topics/api
% https://aws.amazon.com/what-is/api/
% https://www.oracle.com/europe/cloud/cloud-native/api-management/what-is-api/

\subsubsection{SOUP API}


\subsubsection{RESTful API}


\subsubsection{Graph API}


\subsubsection{gRPC API}

\subsection{Backend ako služba}


\subsection{}


\section{Tvorba použivateľského rozhrania}



\subsection{Základné prvky použivateľského rozhrania}
Štvrtou heuristikou Jakoba Nielsna je udržanie konzistencie a štandardy pri návrhu používateľského rozhrania.
Z tohto dôvodu ustálili základné prvky, ktoré sa používajú v rámci používateľského rozhrania.

%https://www.nngroup.com/articles/ui-elements-glossary/


\subsection{Prototyp}
Prototyp slúži na otestovanie použiteľnosti použivateľského rozhrania pred tým ako investujeme prostriedky do finálneho riešenia.
Podľa vernosti s ktorou sa podobajú na finálny produkt rozlišujeme dve kategórie prototypov. Vernosť môže byť v rôznych oblastiach interaktivita, vizuály alebo obsah a príkazy.
Existujú dva druhy prototypov podľa množstva prostriedkov potrebných na ich zostavenie.

\begin{description}
    \item [Prototyp s nízkou vernosťou] {
        umožňuje jednoduché úpravy vďaka jeho jednoduchosti. Jednoduchosť prototypu umožňuje predstavenie základnej štruktúry používateľského rozhrania. Na druhú stranu tento prototyp neposkytuje žiadne takmer žiadnu interaktivitu pre používateľa a všetky prechody sa dejú počas testovania manuálne. 
    }
    \item [Prototyp s vysokou vernosťou] {
        poskytuje reálnejšie odozvy systému. Pri testovaní prototypu s vysokou vernosťou môže nastať nedorozumenie, keď sa môže výsledný systém zdať pripravený na použitie aj keď sa jedná len o prototyp.
    }
\end{description}
% https://www.nngroup.com/articles/ux-prototype-hi-lo-fidelity/



\subsection{Persóny}
https://www.nngroup.com/articles/personas-study-guide/



\section{Testovanie použivateľského rozhrania}


\subsection{Testovania mladistvým}
https://www.nngroup.com/articles/usability-testing-minors/
https://www.nngroup.com/articles/usability-of-websites-for-teenagers/
https://www.nngroup.com/articles/childrens-websites-usability-issues/
https://www.nngroup.com/articles/kids-cognition/

