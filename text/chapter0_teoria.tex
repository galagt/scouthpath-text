\chapter{Teoretický základ}

\section{Softvérové inžinierstvo}

\subsection{Životný cyklus vývoja softvéru SDLC}
Výber správnej metodológie pre SDLC kladie dôležitú rolu na úspech projektu. SDLC metodológie ponúkajú kostru ktorá sprevádza projekt naprieč návrhom, vývojom a implementáciou softvérového riešenia. SDLC je proces ktorý načŕta fázy a aktivity zapojené do tvorby softvéru od kenceptu cez development po údržbu. Medzi typické fázy SDLC patrí:
\begin{description}
    \item[Plánovanie] definuje rozsah a požiadavky projektu, indetifikuje kľúčových členov a ich role a vytvára plán projektu.
    \item[Analýza] získava a okumentuje podrobné požiadavky od používateľov a kľúčových členov. Analyzuje zozbierané potreby na pochopenie funkčných a nefunkčných požiadavkov systému.
    \item[Dizajn] vytvára architektúru aplikácie, definujúc vzájomné interakcie komponent. Navrhuje podobu používateľského rozhrania a používateľských skúsenosti.
    \item[Implementácia] obsahuje vlastnú tvorbu kódu podľa výstupov z predchádzajúcich fáz napr., požiadavky, návrh UI a ďalších.
    \item[Testovanie] prebieha viacerých úrovniach od testovania jednotlivých častí až po systémové testy, ktoré testujú systém ako celok. V rámci testovania sa verifikuje splnenie požiadaviek.
    \item[Nasadenie] prebieha primárne v dvoch fázach, nasadenie do testovacieho prostredia a do ostrého prostredia. V rámci nasadenia sa rieši finálne prostredie v ktorom budú softvér využívať používatelia.
    \item[Údržba a podpora] sa zameriava na monitorovanie a údržbu softvéru v produkčnom prostredí. Prebiehajú v nej opravy na chyby ktoré sa nenašli v rámci predchádzajúcich fáz. Prípadne sem môžu patriť vylepšenia na základe spätnej väzby od používateľov.
\end{description}

Metodológie SDLC sa líšia primárne v prístupe k týmto fázam. Medzi najnámejšie patrí vodopádová metóda, špirálová metóda, iteratívna metóda, inkrementálna metóda, prototypová metóda, metóda v tvare V, metóda rýchlej aplikácie a agilné metódy.

Vodopád je typický lineárnym priechodom naprieč fázami, kde každá fáza začína po skončení predchádzajúcej. Tento prístup je vhodný na projekty s dobre definovanými požiadavkami na začiatku projektu, a prápadné zmeny sú len minimálné. Pevná štruktúra tejto metódy sa ťažko prispôsobuje zmenám po skončení fáze. Výhodou tejto metódy je dobrá odhadnuteľnosť prostriedkov a času na projekt.

Špiralová metóda rozdeľuje projekt do série iterativných cyklov. V každom cykle sa postupne prejde cez všetky fáze. Na konci cyklu sa spraví vyhodnotenie a spolu so spätnou väzbou vstupujú do plánovania ďalšieho cyklu. Toto umožňuje vyššiu flexibiltu na zmeny. Hlavnou vlastnosťou špirálového modelu je jeho zemranie sa na analýzu rizík. Nevýhodou špirálového modelu je jeho komplexnosť a náročnosť na prostriedky. Zároveň mu chýbajú dobre stavené ciele oproti vodopádovej metóde.

Itertívna metóda sa zameriava na postupný vývoj a zdokonalenie softvéru pomocou iterativných cyklov. Umožňuje kontinuálne zlepšovanie a adaptáciu na zmenu požiadavkov. Výhodou je vytvorenie čiastočného no funkčného produktu na konci každej iterácie. Nevýhodou však môže byť prerastanie rozsahu projektu počas vývojového procesu čo môže viesť k prekročeniu rozpočtu a neskorým časom dodania.

Inkrementálna metóda rozdelí na začiatku projekt na ucelené celky. Tieto celky sú následne vývíjané nezávisle a inrementy sú sekvenčne integrované do projektu. Hlavnou výhodou je možnosť paralalelizmu, no toto zároveň prináša väčsie nároky na koordinovanie tímu. Táto metóda je vhodná len na projekty ktoré umožňujú rozdelenie na nezávislé časti.

Prototypová metóda je charakterizovaná čo najrýchlejším vytvorením prototypuna získanie spätnej väzby od používateľov. Prototypová metóda obsahuje takisto viacero iterácií, kde spätná väzba z predchádzajúcej iterácie pomáha dotvárať požiadavky na ďalšiu iteráciu. Pravidelná spätná väzba prispieba k zlepšeniu komunikácie medzi vývojovým tímom a koncovými používateľmi. Medzi nevýhody patrí náročný odhad rozpočtu a dĺžky práce, s neznámim počtom iterácií vopred. Zároveň môžu používatelia nadobudnúť pocit že sa jedná o hotový systém aj keď v realite ide iba o prototyp.

Metóda v tvare V je nadstavbou nad klasickou vodopádovou metódou. Oproti vodopádovej metóde obsahuje proces verifikácie a validácie ktorý je aktívny paralelne s hlavnými fázami vodopádovej metódy. Každá vývojová fáza má prislúchajúcu testovaciu fázu a až po skončení oboch sa prechádza do ďalšej fáze. Takto nastavený proces znižuje riziko vynechania nejakej veci pred pokračovaním vo vývoji.

Metóda rýchlej aplikácie je druh inkrementálnej metódy so zameraním sa na čo najrýchlejšie vytvorenie aplikácie. Často sú používané nástroje na tvorbu kódu a vlastný kód je držaný na minime. Zároveň sa zameriava na modularizáciu softvéru a paralelný vývoj nezávislých modulov čo rapídne znižuje dobu vývoja. Táto metóda je vhodná pre väčšie tími kde je dostatok vývojarov na paralelný vývoj softvéru.

Agilná metodológia reprezentuje rodinu iteratívnych a inkrementálnych metód ktoré prioritizujú adaptabilitu a sploprácu so zákazníkom. V základe rozdeľuje projekt na šprinty. Jeden šprint trvá typicky 1 až 4 týždne a predchádza mu scrum, čo je stretnutie na synchronizáciu vykonávanej práce. Agilný prístup zdieľa väčšinu výhod a nevýhod z ostatných iteratívnych metód.



\subsection{Zber požiadaviek}
Softvérové požiadavky predstavujú spôsob akým môžme zachytiť potreby zákazníka. Upresňujú nám rozsah projektu ako do funkcionilty ktorú zákazník očakáva, tak aj do formy akou budú tie funkcionality využité. Zároveň nám môžu upresňovať rozsah aj z opačnej strany v zmysle funkcionalít a vlastností ktoré softvér obsahovať nebude. Dotvára to celkový obraz zákazníkovi a predchádza sa pomocou toho nedorozumeniam. 

Dôležitou úlohou na začiatku projeku je získavanie a dokumentácia požiadaviek. 
% https://www.indeed.com/career-advice/career-development/requirement-gathering-techniques
% https://medium.com/@marcus_davis/reliable-requirement-gathering-techniques-in-system-analysis-and-design-you-should-know-c28928dfced1


\section{Webové inžinierstvo}

\subsection{Klient-server architektúra}
Klient-server architektúra je sieťový model v ktorom dve hlavné entity, klienti a server, komunikujú medzi sebou za účeľom splenenie špecifickych úloh alebo zdieľania dát. Klient iniciuje požiadavku, čaká na odpoveď serveru a následne zobrazí dáta používateľovi. Server spracúva požiadavky, získa potrebné dáta a pošle ich klientovi. 

Výhodou tejto architektúry je centralizovaná správa prostriedkov a dát. Táto výhoda sa zároveň stáva aj nevýhodou v podobe jediného bodu zlyhania. V prípade zlyhania serveru dochádza k nedostupnosti prostriedkov a dát.

Architekúra môže byť rozšírená o ďalšie prvky medzi serverom a klientom v podobe middlewaru.
%https://medium.com/nerd-for-tech/client-server-architecture-explained-with-examples-diagrams-and-real-world-applications-407e9e04e2d1


%https://www.coursera.org/articles/client-server-architecture
%https://www.ebsco.com/research-starters/architecture/client-server-architecture


\subsection{API}
%https://www.ibm.com/think/topics/api
% https://aws.amazon.com/what-is/api/
% https://www.oracle.com/europe/cloud/cloud-native/api-management/what-is-api/

\subsubsection{SOUP API}


\subsubsection{RESTful API}


\subsubsection{Graph API}


\subsubsection{gRPC API}

\subsection{OAuth}

\subsection{Backend ako služba}


\subsection{}


\section{Tvorba použivateľského rozhrania}



\subsection{Základné prvky použivateľského rozhrania}
Štvrtou heuristikou Jakoba Nielsna je udržanie konzistencie a štandardy pri návrhu používateľského rozhrania.
Z tohto dôvodu ustálili základné prvky, ktoré sa používajú v rámci používateľského rozhrania.

%https://www.nngroup.com/articles/ui-elements-glossary/


\subsection{Prototyp}
Prototyp slúži na otestovanie použiteľnosti použivateľského rozhrania pred tým ako investujeme prostriedky do finálneho riešenia.
Podľa vernosti s ktorou sa podobajú na finálny produkt rozlišujeme dve kategórie prototypov. Vernosť môže byť v rôznych oblastiach interaktivita, vizuály alebo obsah a príkazy.
Existujú dva druhy prototypov podľa množstva prostriedkov potrebných na ich zostavenie.

\begin{description}
    \item [Prototyp s nízkou vernosťou] {
        umožňuje jednoduché úpravy vďaka jeho jednoduchosti. Jednoduchosť prototypu umožňuje predstavenie základnej štruktúry používateľského rozhrania. Na druhú stranu tento prototyp neposkytuje žiadne takmer žiadnu interaktivitu pre používateľa a všetky prechody sa dejú počas testovania manuálne. 
    }
    \item [Prototyp s vysokou vernosťou] {
        poskytuje reálnejšie odozvy systému. Pri testovaní prototypu s vysokou vernosťou môže nastať nedorozumenie, keď sa môže výsledný systém zdať pripravený na použitie aj keď sa jedná len o prototyp.
    }
\end{description}
% https://www.nngroup.com/articles/ux-prototype-hi-lo-fidelity/



\subsection{Persóny}
Persońa je fiktívny, ale realistický popis typického používateľa produktu.


% https://www.nngroup.com/articles/personas-study-guide/



\section{Testovanie použivateľského rozhrania}


\subsection{Testovania mladistvým}
https://www.nngroup.com/articles/usability-testing-minors/
https://www.nngroup.com/articles/usability-of-websites-for-teenagers/
https://www.nngroup.com/articles/childrens-websites-usability-issues/
https://www.nngroup.com/articles/kids-cognition/

